\chapter{搭建本地镜像}\label{chapter:mirror}

\section{查看源使用帮助}

对于Debian 8.x版本来说,163源使用帮助如下。
\begin{Verbatim}[]
deb http://mirrors.163.com/debian/ jessie main non-free contrib
deb http://mirrors.163.com/debian/ jessie-updates main non-free contrib
deb http://mirrors.163.com/debian/ jessie-backports main non-free contrib
deb-src http://mirrors.163.com/debian/ jessie main non-free contrib
deb-src http://mirrors.163.com/debian/ jessie-updates main non-free contrib
deb-src http://mirrors.163.com/debian/ jessie-backports main non-free contrib
deb http://mirrors.163.com/debian-security/ jessie/updates main non-free contrib
deb-src http://mirrors.163.com/debian-security/ jessie/updates main non-free contrib
\end{Verbatim}

其中,jessie-backports部分可以不用下载,关于各个部分的意义见%
“有道笔记”中《Debian、Ubuntu 源列表说明》。

\section{下载源}
安装好apt-mirror工具之后,配置/etc/apt/mirror.list文件:
\begin{Verbatim}[]
set base_path    /media/sf_share/Debian8.x_amd64_mirror

set mirror_path  $base_path/mirror
set skel_path    $base_path/skel
set var_path     $base_path/var
set cleanscript  $var_path/clean.sh
set defaultarch  <running host architecture>
# set postmirror_script $var_path/postmirror.sh
# set run_postmirror 0
set nthreads     20
set _tilde 0

deb-amd64 http://mirrors.163.com/debian/ jessie main non-free contrib
deb-amd64 http://mirrors.163.com/debian/ jessie-updates main non-free contrib
deb-amd64 http://mirrors.163.com/debian/ jessie-backports main non-free contrib
deb-amd64 http://mirrors.163.com/debian-security/ jessie/updates main non-free contr
ib

#clean http://ftp.cn.debian.org/debian
\end{Verbatim}
之后运行:apt-mirror命令即可。对Debian 8.x amd64来说,大概有70G以上的内容,可用tar工具%
将其打包以便拷贝。

\section{配置源}
修改/etc/apt/sources.list文件:
\begin{Verbatim}[]
deb file:/xxx/Debian8.x_amd64_mirror/mirror/mirrors.163.com/debian/           \
    jessie main contrib non-free
deb file:/xxx/Debian8.x_amd64_mirror/mirror/mirrors.163.com/debian/           \
    jessie-updates main non-free contrib
deb file:/xxx/Debian8.x_amd64_mirror/mirror/mirrors.163.com/debian/           \
    jessie-backports main non-free contrib
deb file:/xxx/Debian8.x_amd64_mirror/mirror/mirrors.163.com/debian-security/  \
    jessie/updates main non-free contrib
\end{Verbatim}
然后执行命令:apt-get update即可。

\begin{quote}\kaishu
\textbf{Tips:}使用本地源执行apt-get update时,如出现“Release file expired”错误,%
执行如下命令进行更新操作:apt-get -o Acquire::Check-Valid-Until=false update。或者%
在/etc/apt/apt.conf.d/建立文件例如取名:10no--check-valid-until,然后写入内容:%
\begin{Verbatim}[]
Acquire::Check-Valid-Until "0";
\end{Verbatim}
\end{quote}

将上述内容保存到文件:sources.list.dat之中,配置过程可以使用脚本完成。
\begin{Verbatim}[]
文件:1.setSourcesList.sh
#!/bin/bash

PATH=/sbin:/usr/sbin:/bin:/usr/bin:/usr/local/sbin:/usr/local/bin; export PATH

echo -e "\nYong's 脚本: 备份 sources.list, 设置本地源\n"

/bin/bash ./backupFile.sh /etc/apt/sources.list
cp sources.list.dat /etc/apt/sources.list

apt-get -o Acquire::Check-Valid-Until=false update && apt-get upgrade
\end{Verbatim}
该脚本调用的backupFile.sh用来备份文件,
\begin{Verbatim}[]
文件:backupFile.sh
#!/bin/bash

PATH=/sbin:/usr/sbin:/bin:/usr/bin:/usr/local/sbin:/usr/local/bin; export PATH

fileName=$1

cp -p $1{,.$(date +%F-%H-%M-%S).backup}
\end{Verbatim}
