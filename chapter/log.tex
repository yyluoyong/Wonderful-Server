\chapter{日志}

系统日志详细的知识参考《鸟哥的Linux私房菜——基础篇》相应的部分。

\section{知识要点}

\subsection{日志文件意义}

\begin{itemize*}
    \item \textbf{/var/log/boot.log:}%
当前这次开机的启动信息。

    \item \textbf{/var/log/cron:}%
与 crontab 日程相关。

    \item \textbf{/var/log/dmesg:}%
开机核心信息。

    \item \textbf{/var/log/lastlog:}%
二进制文件,记录系统上面所有的账号最近一次登入系统时的相关信息。使用lastlog命令读取。

    \item \textbf{/var/log/maillog} 或 \textbf{/var/log/mail/*:}%
邮件相关日志。

    \item \textbf{/var/log/messages:}%
几乎系统发生的错误讯息 (或者是重要的信息) 都会记录在这个
档案中。

    \item \textbf{/var/log/secure:}
基本上,只要牵涉到『需要输入账号密码』的软件,那么当登入时 (不管登入正确或错
误) 都会被记录在此档案中。 包括系统的 login 程序、图形接口登入所使用的 gdm 程
序、 su, sudo 等程序、还有网络联机的 ssh, telnet 等程序, 登入信息都会被记载
在这里。

    \item \textbf{/var/log/wtmp}, \textbf{/var/log/faillog}:
二进制文件,这两个档案可以记录正确登入系统者的帐户信息 (wtmp) 与错误登入时所使用的帐户信
息 (faillog) ! 使用last命令读取。

    \item \textbf{/var/log/httpd/*}, \textbf{/var/log/samba/*}:
不同的网络服务会使用它们自己的登录档案来记载它们自己产生的各项讯息!上述的目
录内则是个别服务所制订的登录档。

\end{itemize*}


\subsection{日志文件格式}

\begin{itemize*}
  \item 事件发生的日期与时间;
  \item 发生此事件的主机名;
  \item 启动此事件的服务名称 (如 systemd, CROND 等) 或指令与函式名称 (如
      su, login..);
  \item 该讯息的实际数据内容。
\end{itemize*}


\subsection{服务名称}
\begin{longtable}{ccp{0.65\columnwidth}}\toprule
\textbf{相对序号}	& \textbf{服务类别} &	\makecell[c]{\textbf{说明}} \endhead\hline
0 & 	kern(kernel)& 	就是核心 (kernel) 产生的讯息,大部分都是硬件侦测以及核心功能的启用;\\\hline
1 & 	user& 	在用户层级所产生的信息,例如后续会介绍到的用户使用 logger 指令来记录登录文件的功能;\\\hline
2 & 	mail& 	只要与邮件收发有关的讯息记录都属于这个;\\\hline
3& 	daemon& 	主要是系统的服务所产生的信息,例如 systemd 就是这个有关的讯息!\\\hline
4& 	auth	& 主要与认证/授权有关的机制,例如 login, ssh, su 等需要账号/密码的咚咚;\\\hline
5& 	syslog& 	就是由 syslog 相关协议产生的信息,其实就是 rsyslogd 这支程序本身产生的信息啊!\\\hline
6& 	lpr& 	亦即是打印相关的讯息啊!\\\hline
7& 	news& 	与新闻组服务器有关的东西;\\\hline
8& 	uucp& 	全名为 Unix to Unix Copy Protocol,早期用于 unix 系统间的程序数据交换;\\\hline
9& 	cron	& 就是例行性工作排程 cron/at 等产生讯息记录的地方;\\\hline
10& 	authpriv& 	与 auth 类似,但记录较多账号私人的信息,包括 pam 模块的运作等!\\\hline
11& 	ftp& 	与 FTP 通讯协议有关的讯息输出!\\\hline
16~23& 	local0 ~ local7& 	保留给本机用户使用的一些登录文件讯息,较常与终端机互动。\\
  \bottomrule
\end{longtable}


\subsection{信息等级}
\begin{longtable}{ccp{0.65\columnwidth}}\toprule

\textbf{等级数值}&	\textbf{等级名称}	&\makecell[c]{\textbf{说明}} \endhead\hline
7	&debug	&用来 debug (除错) 时产生的讯息数据;\\\hline
6&	info	&仅是一些基本的讯息说明而已;\\\hline
5&	notice	&虽然是正常信息,但比 info 还需要被注意到的一些信息内容;\\\hline
4&	warning
(warn)	&警示的讯息,可能有问题,但是还不至于影响到某个 daemon 运作的信息;基本上, info, notice, warn 这三个讯息都是在告知一些基本信息而已,应该还不至于造成一些系统运作困扰;\\\hline
3&	err
(error)	&一些重大的错误讯息,例如配置文件的某些设定值造成该服务服法启动的信息说明, 通常藉由 err 的错误告知,应该可以了解到该服务无法启动的问题呢!\\\hline
2&	crit	&比 error 还要严重的错误信息,这个 crit 是临界点 (critical) 的缩写,这个错误已经很严重了喔!\\\hline
1&	alert	&警告警告,已经很有问题的等级,比 crit 还要严重!\\\hline
0&	emerg
(panic)	&疼痛等级,意指系统已经几乎要当机的状态! 很严重的错误信息了。通常大概只有硬件出问题,导致整个核心无法顺利运作,就会出现这样的等级的讯息吧!\\
  \bottomrule
\end{longtable}

\subsection{logrotate配置文件}
\begin{itemize*}
  \item /etc/logrotate.conf
  \item /etc/logrotate.d/
\end{itemize*}

\section{Debian 8.x 日志}

Debian 8.x系统的/etc/rsyslog.conf文件内容如下:

\begin{Verbatim}[]
$ActionFileDefaultTemplate RSYSLOG_TraditionalFileFormat

$FileOwner root
$FileGroup adm
$FileCreateMode 0640
$DirCreateMode 0755
$Umask 0022

$WorkDirectory /var/spool/rsyslog

$IncludeConfig /etc/rsyslog.d/*.conf

auth,authpriv.*			/var/log/auth.log
*.*;auth,authpriv.none		-/var/log/syslog
daemon.*			-/var/log/daemon.log
kern.*				-/var/log/kern.log
lpr.*				-/var/log/lpr.log
mail.*				-/var/log/mail.log
user.*				-/var/log/user.log

mail.info			-/var/log/mail.info
mail.warn			-/var/log/mail.warn
mail.err			/var/log/mail.err

news.crit			/var/log/news/news.crit
news.err			/var/log/news/news.err
news.notice			-/var/log/news/news.notice

*.=debug;\
	auth,authpriv.none;\
	news.none;mail.none	-/var/log/debug
*.=info;*.=notice;*.=warn;\
	auth,authpriv.none;\
	cron,daemon.none;\
	mail,news.none		-/var/log/messages

*.emerg				:omusrmsg:*


daemon.*;mail.*;\
	news.err;\
	*.=debug;*.=info;\
	*.=notice;*.=warn	|/dev/xconsole
\end{Verbatim}

这里需要注意的是:\textbf{/var/log/auth.log}记录了用户认证登录的敏感信息,而不是%
/var/log/secure文件。


\section{保护日志文件}
若系统被入侵,/var/log/目录可能会被恶意删除。可以采取将日志文件定时备份到远程系统%
等措施,防止查找不到攻击源。另外,可以将bash的history记录备份。
